\documentclass{article}
\usepackage{amsfonts}
\usepackage{amsmath}
\usepackage{listings}
\usepackage{mybigpackage}
\usepackage{graphicx}

\graphicspath{{plots/}}

\begin{document}
	\title{\textbf{Assignment-}}
	\author{Abheek Ghosh \\ 
		140123047 }
	
	\maketitle
	

\section{Question A}

\noindent{Code for R}

\begin{lstlisting}[language=R]
rm(list = ls())
p = 0.3;
alpha = 0.05;

n = 1000;
M = c(20, 50, 100, 500, 1000);

Ps = seq(0.001, 1-0.001, 0.001);
z = qchisq(1-alpha, 1);

for (m in M) {
	covProb = 0;
	crossL = 0;
	crossU = 0;

	X = rbinom(m*n, size = 1, prob = p);
	p_0 = sum(X)/(n*m);
	# print(p_mle); print(length(X))

	for (i in 1:n) {
		Y = X[((i-1)*m + 1) : (i*m)];

		p_mle = mean(Y);
		LogL_mle = m*p_mle*log(p_mle) + m*(1-p_mle)*log(1-p_mle);
		LogL_Ps = m*p_mle*log(Ps) + m*(1-p_mle)*log(1-Ps);
		temp = 2*(LogL_mle - LogL_Ps) < z;

		cil = min(Ps[temp]);
		ciu = max(Ps[temp]);

		if (!is.na(cil) || !is.na(ciu)) if ((cil <= p_0) && (p_0 <= ciu)) {
			covProb = covProb + 1;
		}
	}

	LogL_0 = m*p_0*log(p_0) + m*(1-p_0)*log(1-p_0);
	LogL_Ps = m*p_0*log(Ps) + m*(1-p_0)*log(1-Ps);
	temp = 2*(LogL_0 - LogL_Ps) < z;
	cil = min(Ps[temp]);
	ciu = max(Ps[temp]);

	cat(sprintf("\nBernoulli sample count = %d\\\\\np_0 = %f\\\\\n", m, p_0));
	cat(sprintf("Using Likelihood Ratio Test Confidence Interval = [%f, %f]\\\\\n", cil, ciu));
	cat(sprintf("Coverage Probability = %f\\\\\n", covProb/n));
}

\end{lstlisting}

Bernoulli sample count = 20\\
p0 = 0.300400\\
Using Likelihood Ratio Test Confidence Interval = [0.133000, 0.516000]\\
Coverage Probability = 0.956000

Bernoulli sample count = 50\\
p0 = 0.297460\\
Using Likelihood Ratio Test Confidence Interval = [0.184000, 0.432000]\\
Coverage Probability = 0.955000

Bernoulli sample count = 100\\
p0 = 0.299070\\
Using Likelihood Ratio Test Confidence Interval = [0.216000, 0.393000]\\
Coverage Probability = 0.958000

Bernoulli sample count = 500\\
p0 = 0.298872\\
Using Likelihood Ratio Test Confidence Interval = [0.260000, 0.339000]\\
Coverage Probability = 0.950000

Bernoulli sample count = 1000\\
p0 = 0.299089\\
Using Likelihood Ratio Test Confidence Interval = [0.272000, 0.327000]\\
Coverage Probability = 0.948000


\section{Question B}

\noindent{Code for R}

\begin{lstlisting}[language=R]
rm(list = ls())
d = read.table("d-csp0108.txt", header=TRUE)
names = c('C', 'SP')
T = length(d[,1]);

# Calculating log returns
for (k in 2:3) {
	d[,k] = log(1 + d[,k]);
}

skew <- function(X) {
	T = length(X);
	mu = mean(X);
	sig = sqrt( sum((X-mu)^2)/(T-1) );
	sk = sum((X-mu)^3)/(T-1) / sig^3;
	return (sk);
}

kurt <- function(X) {
	T = length(X);
	mu = mean(X);
	sig = sqrt( sum((X-mu)^2)/(T-1) );
	kt = sum((X-mu)^4)/(T-1) / sig^4;
	return (kt);
}

alpha = 0.05;
type = c('Skewness', 'Excess Kurtosis');

for (k in 2:3) {
	for (ty in type) {
		X = d[,k];

		if (ty == 'Skewness') {
			theta = skew(X);
			sig = sd(X)*sqrt(6/T);
			clb = sig*qnorm(alpha/2); cub = sig*qnorm(1 - alpha/2);
		} else {
			theta = kurt(X) - 3;
			sig = sd(X)*sqrt(24/T);
			clb = sig*qnorm(alpha/2); cub = sig*qnorm(1 - alpha/2);
		}
		

		cat(sprintf('\n%s Stock\\\\\n', names[k-1]));
		cat(sprintf('The %d%% confidence interval for %s = [%f, %f]\\\\\n', 100*(1-alpha), ty, clb, cub));
		if ((clb <= theta) && (theta <= cub)) {
			cat(sprintf('The %s = %f is inside the confidence interval.\\\\\n\n', ty, theta));
		} else {
			cat(sprintf('The %s = %f is not inside the confidence interval.\\\\\n\n', ty, theta));
		}
	}
}

\end{lstlisting}

C Stock\\
The 95\% confidence interval for Skewness = [-0.107058, 0.107058]\\
The Skewness = 0.538650 is not inside the confidence interval. Hypothesis False\\


C Stock\\
The 95\% confidence interval for Excess Kurtosis = [-0.214115, 0.214115]\\
The Excess Kurtosis = 42.760151 is not inside the confidence interval. Hypothesis False\\


SP Stock\\
The 95\% confidence interval for Skewness = [-0.107058, 0.107058]\\
The Skewness = -0.140850 is not inside the confidence interval. Hypothesis False\\


SP Stock\\
The 95\% confidence interval for Excess Kurtosis = [-0.214115, 0.214115]\\
The Excess Kurtosis = 9.956392 is not inside the confidence interval. Hypothesis False\\


\end{document}
