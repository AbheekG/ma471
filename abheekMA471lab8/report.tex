\documentclass{article}
\usepackage{amsfonts}
\usepackage{amsmath}
\usepackage{listings}
\usepackage{mybigpackage}
\usepackage{graphicx}

\graphicspath{{plots/}}

\begin{document}
	\title{\textbf{Assignment-8}}
	\author{Abheek Ghosh \\ 
		140123047 }
	
	\maketitle
	

\section{Question 1}

\noindent{Code for R}

\begin{lstlisting}[language=R]
rm(list = ls())
d = read.table("FRWRD.txt", header=FALSE);

cvd_error1 <- function(X, Y, n, r) {
	s = 0;

	for (i in 1:n) {
		X1 = X[-i,];
		Y1 = as.matrix(Y[-i,]);
		# print(X1);
		# print(Y1);
		fn <- function(beta) {
			return (sum((Y-X %*% beta)^2));
		}
		# beta = solve(t(X1) %*% X1) %*% (t(X1) %*% Y1);
		beta = optim(runif(r), fn)$par;
		# print(beta);
		s = s + sum((Y-X %*% beta)^2);
	}
	return (s/n);
}

cvd_error2 <- function(X, Y, n, r) {
	s = 0;
	k = 3;
	m = 2*n;

	for (i in 1:m) {
		leave = -sample.int(n, k);
		X1 = X[leave,];
		Y1 = as.matrix(Y[leave,]);
		# print(X1);
		# print(Y1);
		fn <- function(beta) {
			return (sum((Y-X %*% beta)^2));
		}
		# beta = solve(t(X1) %*% X1) %*% (t(X1) %*% Y1);
		beta = optim(runif(r), fn)$par;
		# print(beta);
		s = s + sum((Y-X %*% beta)^2);
	}
	return (s/m);
}

Y = d[,1];
n = length(Y);
Y = matrix(Y, nrow = n, ncol = 1);
# print(Y);

R = c(3, 6, 8);
mincv1 = numeric();
mincv2 = numeric();

for (r in R) {
	a = numeric();
	for (i in 0:r) {
		a = c(a, (1:n)^i);
	}
	X = matrix(a, nrow = n, ncol = r+1);
	mincv1 = c(mincv1, cvd_error1(X, Y, n, r+1));
	mincv2 = c(mincv2, cvd_error2(X, Y, n, r+1));
	# print(a);
}

print('Error LOOCV then k-fold');
print(mincv1); print(mincv2);
print('degree with minimum error by LOOCV')
print(R[which.min(mincv1)]);

print('degree with minimum error by k-fold')
print(R[which.min(mincv2)]);
\end{lstlisting}

"Error LOOCV then k-fold"\\
1.248019e+03 2.919813e+13 1.949899e+19\\
1.606876e+04 1.951479e+13 1.587205e+19\\

"degree with minimum error by LOOCV"\\
3\\

"degree with minimum error by k-fold"\\
3

\section{Question 2}

\noindent{Code for R}

\begin{lstlisting}[language=R]
rm(list = ls())

r = 7;
n = 16;

Y = as.matrix(longley[1:n, r]);
X = as.matrix(longley[1:n, 1:(r-1)]);

X = cbind(rep(1, n), X);

beta = solve(t(X) %*% X) %*% (t(X) %*% Y);
print('Best fit');
print(beta[,1]);

# Outliers
xm = colSums(X)/n;

print('Most Outlier');
print(X[which.max(colSums((t(X) - xm)^2) + (Y - X %*% beta)^2), ])

# Leverage point
print('Most Influential Point');
print(X[which.max((Y - X %*% beta)^2), ])


print('Most Leverage point');
print(X[which.max(colSums((t(X) - xm)^2)), ])

\end{lstlisting}

"Most Outlier"\\
             GNP.deflator          GNP   Unemployed Armed.Forces   Population	Year \\
       1.000      115.700      518.173      480.600      257.200      127.852   1961.000 

"Most Influential Point"\\
             GNP.deflator          GNP   Unemployed Armed.Forces   Population	Year \\
       1.000      104.600      419.180      282.200      285.700      118.734	1956.000 
        
     
"Most Leverage point"\\
             GNP.deflator          GNP   Unemployed Armed.Forces   Population 	Year \\
       1.000      115.700      518.173      480.600      257.200      127.852 	1961.000
        

\end{document}
