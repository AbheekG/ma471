\documentclass{article}
\usepackage{amsfonts}
\usepackage{amsmath}
\usepackage{listings}
\usepackage{mybigpackage}
\usepackage{graphicx}

\graphicspath{{plots/}}

\begin{document}
	\title{\textbf{Assignment-}}
	\author{Abheek Ghosh \\ 
		140123047 }
	
	\maketitle
	

\section{Question a}

\noindent{Code for R}

\begin{lstlisting}[language=R]
rm(list = ls())
d = read.table("d-csp0108.txt", header=TRUE)

# Calculating log returns
for (k in 2:3) {
	d[,k] = log(1 + d[,k]);
	# print(mean(d[,k]));
}
\end{lstlisting}

\pagebreak


\section{Question i}

\noindent{Code for R}

\begin{lstlisting}[language=R]
rm(list = ls())
d = read.table("d-csp0108.txt", header=TRUE)
names = c('C', 'SP')
# Calculating log returns
for (k in 2:3) {
	d[,k] = log(1 + d[,k]);
	cat(names[k-1], "data, Mean =", mean(d[,k]), "\\\\\n");
}

cat("Yes, mean different from 0.\\\\\n")

alpha = 0.05;
n = length(d[,1]);

for (k in 2:3) {
	mu = mean(d[,k]);
	sig = sd(d[,k]);

	# X = (d[,k] - mu)/sig;
	X = d[,k];
	# covProb = sum((qnorm(alpha/2) <= X) & (X <= qnorm(1 - alpha/2)));
	covProb = sum( ((X + sig*qnorm(alpha/2))/(1+sig) <= mu) & (mu <= (X + sig*qnorm(1 - alpha/2))/(1+sig)) );
	cat(names[k-1], "data, Coverage Probability =", covProb/n, "\\\\\n")
}
\end{lstlisting}

C data, Mean = -0.0008447986 \\
SP data, Mean = -0.0001887621 \\
Yes, mean different from 0.\\
C data, Coverage Probability = 0.9597215 \\
SP data, Coverage Probability = 0.9507708 \\

\pagebreak

\section{Question b}

\noindent{Code for R}

\begin{lstlisting}[language=R]
rm(list = ls())
p = 0.3;
alpha = 0.05;

n = 1000;
M = c(20, 50, 100);

for (m in M) {
	covProb = 0;
	crossL = 0;
	crossU = 0;

	X = rbinom(m*n, size = 1, prob = p);
	p_mle = sum(X)/(n*m);
	# print(p_mle); print(length(X))

	for (i in 1:n) {
		Y = X[((i-1)*m + 1) : (i*m)];

		p_mean = mean(Y);
		# print((i-1)*m); print(p_mean)
		err = -qnorm(alpha/2)*sqrt(p_mean*(1-p_mean)/m);
		cil = p_mean - err;
		ciu = p_mean + err;

		if (cil < 0) {
			cil = 0;
			crossL = crossL + 1;
		}

		if (ciu > 1) {
			ciu = 1;
			crossU = crossU + 1;
		}

		if ((cil <= p_mle) && (p_mle <= ciu)) {
			covProb = covProb + 1;
		}
	}
	err = -qnorm(alpha/2)*sqrt(p_mle*(1-p_mle)/m);

	cat(sprintf("\nBernoulli sample count = %d\np_mle = %f\n", m, p_mle));
	cat(sprintf("Confidence Interval = [%f, %f]\n", p_mle-err, p_mle+err));
	cat(sprintf("Coverage Probability = %f\n", covProb/n));
	cat(sprintf("Number of instances of interval LOWER bound going outside parameter space = %d\n", crossL));
	cat(sprintf("Number of instances of interval UPPER bound going outside parameter space = %d\n", crossU));
}
\end{lstlisting}

Truncating the confidence interval from 0 is only partially justified. If we use central limit theorem and consider many IID samples then it approximates to normal distribution, but it doesn't fully become so for less number of samples of distributions other than normal. Distribution is bernoulli. It is evident from our result also: for 20 samples sometimes the interval goes beyond feasible region. As we are already doing approximation by using CLT for small samples, we can justify truncating the interval, it will not decrease the probability (confidence) of the interval. Other modifications can be by changing the coverage probability.

Bernoulli sample count = 20\\
p mle = 0.300900\\
Confidence Interval = [0.099892, 0.501908]\\
Coverage Probability = 0.938000\\
Number of instances of interval LOWER bound going outside parameter space = 112\\
Number of instances of interval UPPER bound going outside parameter space = 0\\

Bernoulli sample count = 50\\
p mle = 0.297940\\
Confidence Interval = [0.171171, 0.424709]\\
Coverage Probability = 0.943000\\
Number of instances of interval LOWER bound going outside parameter space = 0\\
Number of instances of interval UPPER bound going outside parameter space = 0\\

Bernoulli sample count = 100\\
p mle = 0.299270\\
Confidence Interval = [0.209516, 0.389024]\\
Coverage Probability = 0.966000\\
Number of instances of interval LOWER bound going outside parameter space = 0\\
Number of instances of interval UPPER bound going outside parameter space = 0\\



\end{document}
