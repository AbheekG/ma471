\documentclass{article}
\usepackage{amsfonts}
\usepackage{amsmath}
\usepackage{listings}
\usepackage{mybigpackage}
\usepackage{graphicx}

\graphicspath{{plots/}}

\begin{document}
	\title{\textbf{Assignment-9}}
	\author{Abheek Ghosh \\ 
		140123047 }
	
	\maketitle
	

\section{Question 1}

\noindent{Code for R}

\begin{lstlisting}[language=R]
rm(list = ls())

d = read.csv('Data_gold.txt', header = FALSE);
d = d[, 'V4'];
# print(d)

# a
d = ts(d);
plot.ts(d, ylab = 'Gold Price', main = 'Gold price vs time');
dev.copy(png, "plots/plot1_a.png")
dev.off ()

# b
X = log(d[-1]) - log(d[-length(d)]);
X_ts = ts(X);
plot.ts(X_ts, ylab="Log returns of Gold", main = "Log returns of Gold vs Time");
dev.copy(png, "plots/plot1_b.png");
dev.off ();

# c
acf(X);
dev.copy(png, "plots/plot1_c.png"); dev.off ();
pacf(X)
dev.copy(png, "plots/plot1_d.png"); dev.off ();

# d
X_ar = ar(X);
print(X_ar);

# e
pred = predict(X_ar, n.ahead = 3);
print(pred);
plot(pred$pred, ylab = "Log returns", main = "3-step Prediction of Log returns");
dev.copy(png, "plots/plot1_e.png"); dev.off ();
\end{lstlisting}

(a)
\includegraphics{"plot1_a"}
\pagebreak

(b)
\includegraphics{"plot1_b"}
\pagebreak

(c)
Yes, there are serial autocorrelations in the r series, as visible from plot.
\includegraphics{"plot1_c"}
\pagebreak

(d)
We create the ar model with p = 25. It is correct because the PACF goes to 0 after lag 25.

Call:
ar(x = X)

Coefficients:
      1        2        3        4        5        6        7        8  
-0.0488   0.0066   0.0121   0.0011   0.0406  -0.0118  -0.0473   0.0210  
      9       10       11       12       13       14       15       16  
 0.0138  -0.0204  -0.0206  -0.0279  -0.0180   0.0243  -0.0091   0.0241  
     17       18       19       20       21       22       23       24  
-0.0002  -0.0194   0.0151   0.0277   0.0001  -0.0369  -0.0203  -0.0173  
     25  
-0.0463

Order selected 25  sigma\^2 estimated as  0.0001093

\includegraphics{"plot1_d"}
\pagebreak

(e)
Prediction

pred
Time Series:
Start = 5876 
End = 5878 
Frequency = 1 
[1] -0.0005876651  0.0001947626  0.0003147521

se
Time Series:
Start = 5876 
End = 5878 
Frequency = 1 
[1] 0.01045254 0.01046498 0.01046541


\includegraphics{"plot1_e"}
\pagebreak


\section{Question 2}

\noindent{Code for R}

\begin{lstlisting}[language=R]
rm(list = ls())

sig = 0.025;
alp = 0.2;

# a
n = 100;
A = rnorm(n = n+1, mean = 0, sd = sig);
X = A[-1] + alp*A[-(n+1)];

# b
acf(X);
dev.copy(png, "plots/plot2_a.png"); dev.off ();

# c
plot(0:4, c(1, 0.2/1.04, 0, 0, 0), type="h", col = "blue");
dev.copy(png, "plots/plot2_b.png"); dev.off ();
\end{lstlisting}

(a)
\includegraphics{"plot2_a"}
\pagebreak

(b)
\includegraphics{"plot2_b"}
\pagebreak

(c)
\includegraphics{"plot2_c"}
\pagebreak


\section{Question 3}

\noindent{Code for R}

\begin{lstlisting}[language=R]
rm(list = ls())

d = read.table('deciles08.txt');
# print(d)
d2 = d[,3];
d10 = d[,5];

# a
acf(d2, main = "Decile 2 ACF vs Lag");
dev.copy(png, "plots/plot3_a.png"); dev.off ();
acf(d10, main = "Decile 10 ACF vs Lag");
dev.copy(png, "plots/plot3_b.png"); dev.off ();

# b
d2_arma = arima(d2, order = c(5, 1, 5));
print(d2_arma)

#3c
arma_forecast = predict(d2_arma, n.ahead = 12);
print(arma_forecast)
plot(arma_forecast$pred, ylab = "Forecasted values")
dev.copy(png, "plots/plot3_c.png"); dev.off ();
\end{lstlisting}

(a)
Decile 2. The hypothesis is false. The ACF does not go to zero after lag 12 at 5\% confidence level.
\includegraphics{"plot3_a"}
\pagebreak

Decile 10. The hypothesis is true. The ACF goes to zero after lag 12 at 5\% confidence level.
\includegraphics{"plot3_b"}
\pagebreak

(b)

Call:
arima(x = d2, order = c(5, 1, 5))

Coefficients:
         ar1      ar2      ar3      ar4     ar5      ma1     ma2      ma3
      0.3634  -0.2544  -0.1020  -0.6196  0.1252  -1.3056  0.6177  -0.1623
s.e.  0.3187   0.2969   0.2526   0.2686  0.0489   0.3205  0.5783   0.5255
         ma4      ma5
      0.4981  -0.6144
s.e.  0.4631   0.2673

sigma\^2 estimated as 17391:  log likelihood = -2950.69,  aic = 5923.38

(c)
pred
Time Series:
Start = 470 
End = 481 
Frequency = 1 
 [1] 195.6041 210.9150 222.6653 227.2416 213.1335 200.2408 193.3154 194.1547
 [9] 206.8511 218.1788 221.6556 217.3543

se
Time Series:
Start = 470 
End = 481 
Frequency = 1 
 [1] 131.8760 132.0959 132.5031 132.7527 132.7932 133.5586 133.5814 133.6028
 [9] 133.6035 133.6492 133.6960 133.9558

\includegraphics{"plot3_c"}
\pagebreak

\end{document}
