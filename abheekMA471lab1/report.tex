\documentclass{article}
\usepackage{amsfonts}
\usepackage{amsmath}
\usepackage{listings}
\usepackage{mybigpackage}
\usepackage{graphicx}

\graphicspath{{plots/}}

\begin{document}
	\title{\textbf{Assignment-1}}
	\author{Abheek Ghosh \\ 
		140123047 }
	
	\maketitle
	

\section{Question 1}

\noindent{Code for R}

\begin{lstlisting}
rm(list = ls())
d = read.table("d-csp0108.txt", header=TRUE)

k = 2
plot(d[,1], d[,k], main="C return vs Time", xlab="Time", ylab="Return")
dev.copy(png, "plots/plota1.png")
dev.off ()
k = 3
plot(d[,1], d[,k], main="SP return vs Time", xlab="Time", ylab="Return")
dev.copy(png, "plots/plota2.png")
dev.off ()

k = 2
hist(d[,k], freq=FALSE, breaks=100, col='green', main='Probability Distribution of C returns', xlab='C returns')
x <- seq(-0.5, 0.5, length=1000)
lines(x, dnorm(x, mean(d[,k]), sd(d[,k])))
dev.copy(png, "plots/plota3.png")
dev.off ()

k = 3
hist(d[,k], freq=FALSE, breaks=100, col='green', main='Probability Distribution of SP returns', xlab='SP returns')
x <- seq(-0.5, 0.5, length=1000)
lines(x, dnorm(x, mean(d[,k]), sd(d[,k])))
dev.copy(png, "plots/plota4.png")
dev.off ()
\end{lstlisting}

\includegraphics{"plota1"}
\pagebreak

\includegraphics{"plota2"}
\pagebreak

\includegraphics{"plota3"}
\pagebreak

\includegraphics{"plota4"}
\pagebreak

\section{Question 2}

\noindent{Code for R}

\begin{lstlisting}
rm(list = ls())
library(MASS)

ddoublex = function(x, mu = 0, lambda = 1) {
	a = abs(x - mu)
	return (dexp(a, lambda)/2)
}

dmixnorm = function(x, p = 0.5, mu1 = 0, mu2 = 0, sig1 = 1, sig2 = 100) {
	return (p*dnorm(x, mu1, sig1) + (1-p)*dnorm(x, mu2, sig2))
}

d = read.table("d-csp0108.txt", header=TRUE)

names = c('C', 'SP')
cols = c('red', 'blue', 'green', 'black')
dists = c('t-distribution', 'double-exponential', 'cauchy', 'mixture of normal')

for (k in 2:3) {
	x <- seq(-0.5, 0.5, length=1000)
	mu_d = mean(d[,k])
	sd_d = sd(d[,k])

	hist(d[,k], prob=T, breaks=100, main = sprintf("Probability Distribution Function (PDF) of %s returns", names[k-1]), 
		xlab = sprintf("%s returns", names[k-1]), ylab = "Probability Density")
	
	
	# t-dist	
	param = fitdistr(d[,k], "t", start = list(m = mu_d, s = sd_d, df=3))#, lower=c(-1, 0.001,1))
	param = param$estimate
	lines(x, dt((x-mu_d)/sd_d, df=param[3])/sd_d, col = cols[1])

	param = fitdistr(d[,k], ddoublex, start = list(mu = 0, lambda = 1))
	param = param$estimate
	# print(param)
	lines(x, ddoublex((x), mu = param[1], lambda = param[2]),  col = cols[2])

	param = fitdistr(d[,k], "cauchy", start = list(location = 0, scale = 1))
	param = param$estimate
	lines(x, dcauchy((x-mu_d), location = param[1], scale = param[2]),  col = cols[3])

	sd1 = sd_d/2
	p = 0.7
	param = c(p, mu_d, mu_d, sd1, ( (sd_d^2 - p*sd1^2) / (1-p) )^0.5)
	# print(param)
	lines(x, dmixnorm(x, p = param[1], mu1 = param[2], mu2 = param[3], sig1 = param[4], sig2 = param[5]),  col = cols[4])

	legend("topright", legend = dists, fill = cols)
	dev.copy(png, sprintf("plots/plotb%d.png", k-1))
	dev.off ()
}
\end{lstlisting}

\includegraphics{"plotb1"}
\pagebreak

\includegraphics{"plotb2"}
\pagebreak

From the plots we observe that the distribution of the given data fits very well with the assumed distribution. The probability distribution function (PDF) and the survival function both fit very well. From the Quantile-Quantile plot we observe that the given data is heavier tailed with respect to double exponential distribution and mixed normal distribution. On the other hand, t-distribution and the cauchy distribution are heavier tailed than the given data.

\section{Question 3}

\noindent{Code for R}

\begin{lstlisting}
rm(list = ls())
library(MASS)

ddoublex = function(x, mu = 0, lambda = 1) {
	a = abs(x - mu)
	return (dexp(a, lambda)/2)
}

dmixnorm = function(x, p = 0.5, mu1 = 0, mu2 = 0, sig1 = 1, sig2 = 100) {
	return (p*dnorm(x, mu1, sig1) + (1-p)*dnorm(x, mu2, sig2))
}

rdoublex = function(n, mu = 0, lambda = 1) {
	D = rexp(n, lambda)
	temp = runif(n)
	D[temp > 0.5] = -D[temp > 0.5]
	D = D + mu
	return(D)
}

rmixnorm = function(n, p = 0.5, mu1 = 0, mu2 = 0, sig1 = 1, sig2 = 100) {
	n1 = as.integer(n*p)
	D1 = rnorm(n1, mu1, sig1)
	D2 = rnorm(n - n1, mu2, sig2)
	D = c(D1, D2)
	return(D)
}

d = read.table("d-csp0108.txt", header=TRUE)

names = c('C', 'SP')
cols = c('red', 'blue', 'green', 'brown')
dists = c('t-distribution', 'double-exponential', 'cauchy', 'mixture of normal')

for (k in 2:3) {
	p = seq(0, 1, 0.01)
	Q_data = quantile(d[,k], probs = p)
	mu_d = mean(d[,k])
	sd_d = sd(d[,k])

	refLine = seq(min(Q_data), max(Q_data), 0.01)
	plot(refLine, refLine, main = sprintf("QQ-plot of %s and Normal", names[k-1]), 
		xlab = "Base distribution Quantile", ylab = sprintf("%s Quantile", names[k-1]), col = 'black', type = 'l')	
	
	# t-dist	
	param = fitdistr(d[,k], "t", start = list(m = mu_d, s = sd_d, df=3))#, lower=c(-1, 0.001,1))
	param = param$estimate
	# lines(x, dt((x-mu_d)/sd_d, df=param[3])/sd_d, col = cols[1])
	Q_t = qt(p, df=param[3])*sd_d + mu_d
	lines(Q_t, Q_data, col = cols[1])

	param = fitdistr(d[,k], ddoublex, start = list(mu = 0, lambda = 1))
	param = param$estimate
	# print(param)
	Q_doublex = quantile(rdoublex(length(p), mu = param[1], lambda = param[2]), probs = p)
	lines(Q_doublex, Q_data, col = cols[2])

	param = fitdistr(d[,k], "cauchy", start = list(location = 0, scale = 1))
	param = param$estimate
	Q_cauchy = quantile(rcauchy(length(p), location = param[1], scale = param[2]), probs = p)
	lines(Q_cauchy, Q_data, col = cols[3])

	sd1 = sd_d/2
	pp = 0.7
	param = c(pp, mu_d, mu_d, sd1, ( (sd_d^2 - pp*sd1^2) / (1-pp) )^0.5)
	Q_mixnorm = quantile(rmixnorm(length(p), p = param[1], mu1 = param[2], mu2 = param[3], sig1 = param[4], sig2 = param[5]), probs = p)
	lines(Q_mixnorm, Q_data, col = cols[4])

	legend("bottomright", legend = dists, fill = cols)
	dev.copy(png, sprintf("plots/plotc%d.png", k-1))
	dev.off ()
}
\end{lstlisting}

\includegraphics{"plotc1"}
\pagebreak

\includegraphics{"plotc2"}
\pagebreak

From the plots we observe that the distribution of the given data fits very well with the assumed distribution. From the Quantile-Quantile plot we observe that the given data is heavier tailed with respect to double exponential distribution and mixed normal distribution. On the other hand, t-distribution and the cauchy distribution are heavier tailed than the given data.

\section{Question 4}

\noindent{Code for R}

\begin{lstlisting}
rm(list = ls())
library(MASS)

ddoublex = function(x, mu = 0, lambda = 1) {
	a = abs(x - mu)
	return (dexp(a, lambda)/2)
}

dmixnorm = function(x, p = 0.5, mu1 = 0, mu2 = 0, sig1 = 1, sig2 = 100) {
	return (p*dnorm(x, mu1, sig1) + (1-p)*dnorm(x, mu2, sig2))
}

rdoublex = function(n, mu = 0, lambda = 1) {
	D = rexp(n, lambda)
	temp = runif(n)
	D[temp > 0.5] = -D[temp > 0.5]
	D = D + mu
	return(D)
}

rmixnorm = function(n, p = 0.5, mu1 = 0, mu2 = 0, sig1 = 1, sig2 = 100) {
	n1 = as.integer(n*p)
	D1 = rnorm(n1, mu1, sig1)
	D2 = rnorm(n - n1, mu2, sig2)
	D = c(D1, D2)
	return(D)
}

d = read.table("d-csp0108.txt", header=TRUE)

names = c('C', 'SP')
cols = c('red', 'blue', 'green', 'brown')
dists = c('t-distribution', 'double-exponential', 'cauchy', 'mixture of normal')

for (k in 2:3) {
	x = seq(min(d[,k]), max(d[,k]), 0.01)
	mu_d = mean(d[,k])
	sd_d = sd(d[,k])
	cdf_d = ecdf(d[,k])

	plot(x, 1- cdf_d(x), main = sprintf("CDF of %s and other distributions", names[k-1]), 
		xlab = "x", ylab = sprintf("%s CDF", names[k-1]), col = 'black', type = 'l')	
	
	# t-dist	
	param = fitdistr(d[,k], "t", start = list(m = mu_d, s = sd_d, df=3))#, lower=c(-1, 0.001,1))
	param = param$estimate
	lines(x, 1-pt((x-mu_d)/sd_d, df=param[3]) , col = cols[1])

	param = fitdistr(d[,k], ddoublex, start = list(mu = 0, lambda = 1))
	param = param$estimate
	cdf_base = ecdf(rdoublex(length(d[,k]), mu = param[1], lambda = param[2]))
	lines(x, 1-cdf_base(x) , col = cols[2])

	param = fitdistr(d[,k], "cauchy", start = list(location = 0, scale = 1))
	param = param$estimate
	cdf_base = ecdf(rcauchy(length(d[,k]), location = param[1], scale = param[2]))
	lines(x, 1-cdf_base(x) , col = cols[3])

	sd1 = sd_d/2
	pp = 0.7
	param = c(pp, mu_d, mu_d, sd1, ( (sd_d^2 - pp*sd1^2) / (1-pp) )^0.5)
	cdf_base = ecdf(rmixnorm(length(d[,k]), p = param[1], mu1 = param[2], mu2 = param[3], sig1 = param[4], sig2 = param[5]))
	lines(x, 1-cdf_base(x) , col = cols[4])

	legend("bottomright", legend = dists, fill = cols)
	dev.copy(png, sprintf("plots/plotd%d.png", k-1))
	dev.off ()
}
\end{lstlisting}

\includegraphics{"plotd1"}
\pagebreak

\includegraphics{"plotd2"}
\pagebreak

From the plots we observe that the distribution of the given data fits very well with the assumed distribution. The probability distribution function (PDF) and the survival function both fit very well. From the Quantile-Quantile plot we observe that the given data is heavier tailed with respect to double exponential distribution and mixed normal distribution. On the other hand, t-distribution and the cauchy distribution are heavier tailed than the given data.
\end{document}
